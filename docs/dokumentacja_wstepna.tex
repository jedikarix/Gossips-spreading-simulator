\documentclass{article}

\usepackage{fancyhdr}
\usepackage[english,polish]{babel}
\usepackage[utf8]{inputenc}
\usepackage{polski}
\usepackage{titling}
\usepackage{graphicx}

\title{Projekt WEDT + SAG \\ Symulator rozchodzenia się plotek \\ Dokumentacja wstępna}
\author{Karol Chęciński, Tomasz Gałecki, Jakub Łyskawa}
\setlength{\droptitle}{-10em}
\begin{document}
	\maketitle
	\section{Temat projektu}
	\section{Opis modelu}
	\subsection{Graf połączeń}
	\begin{itemize}
		\item{Cechy grafu}
		\item{Elementy grafu}
		\item{Topologie}
	\end{itemize}
	\subsection{Charakterystyka agenta}
	Każdy węzeł w grafie połączeń reprezentuje agenta.
	\begin{description}
		\item[Interakcje między agentami]
		Agent może otrzymywać oraz przekazywać informacje innym agentom.
		Protokołem komunikacyjnym agentów jest język naturalny, tj.
		przekazanie informacji między agentami następuje poprzez rozmowę
		w języku angielskim zakodowaną tekstowo.
		\item[Stan agenta]
		Stan agenta składa się z trzech elementów:
		\begin{itemize}
			\item {Informacje otrzymane od innych agentów - 
			Liczba tych informacji może być ograniczona lub nie, co jest sterowane parametrem,
			ponadto nie są duplikowane informacje które agent już posiada albo sprzeczne z posiadanymi.}
			\item {Poziom zaufania do każdej posiadanej informacji}
			\item {Poziom zaufania do każdego połączonego agenta}
		\end{itemize}
		\item[Model wiedzy]
		Wiedza agenta opisuje w jaki sposób są modyfikowane otrzymane przez niego informacje.\\
		Początkowa wiedza agenta nie zależy od jego stanu.
		Otrzymane przez agenta informacje traktowane są jako jego dodatkowa wiedza,
		nakładająca się na pewną wiedzę początkową.\\
		Wiedza reprezentowana jest przy użyciu n-gramów,
		gdzie wiedza początkowa zostanie utworzona korzystając z pewnego korpusu,
		a n-gramy wynikające ze stanu agenta uwzględniane są z pewną wagą.
		\item[Parametry agenta] Każdy agent może być niezależnie parametryzowany następującymi wartościami:
		\begin{itemize}
			\item{Bazowe prawdopodobieństwo przekazania dalej otrzymanej informacji}
			\item{Maksymalna liczba przechowywanych informacji}
			\item{Bazowy wpływ informacji na wiedzę}
			\item{Ufność}
			\item{Parametry związane z modyfikacją informacji}
		\end{itemize}
	\end{description}
	\subsection{Model zaufania}
	Wstępny opis działania modelu zaufania:
	\begin{enumerate}
		\item Agent otrzymuje od innego agenta $n$ informację $p$
		\item Poziom zaufania informacji $p$: $ z = sigmoid(U \cdot Z_a(n)) $
		\item Jeżeli to pierwsza otrzymana informacja,
		zapamiętaj $p$ z $ Z_p(p) \leftarrow z, A(p) \leftarrow n $.
		W przeciwnym wypadku kontynuuj.
		\item Wybierz z przechowywanych informację $p'$ najsilniej związaną semantycznie z $p$
		\item Jeżeli $p'$ jest niezwiązana z $p$, zapamiętaj $p$ z $ Z_p(p) \leftarrow z, A(p) \leftarrow n $ i przerwij.
		\item Jeżeli $p'$ jest równoznaczna z $p$ to:\\
		$ Z_p(p') \leftarrow Z_p(p') + z, Z_a(A(p')) \leftarrow Z_a(A(p')) + z , A(p') \leftarrow n $
		\item Jeżeli $p'$ jest sprzeczna z $p$ to:\\
		$ Z_p(p') \leftarrow Z_p(p') - z, Z_a(A(p')) \leftarrow Z_a(A(p')) - z$\\
		Jeżeli $ Z_p(p') < 0 $ to zapomnij $p'$, zapamiętaj $p$ z\\
		$ Z_p(p) \leftarrow -Z_p(p'), A(p) \leftarrow n $
	\end{enumerate}
	Oznaczenia:
	\begin{itemize}
		\item[$Z_a(x)$] Poziom zaufania agenta do agenta $x$, początkowo $0$
		\item[$Z_p(x)$] Poziom zaufania agenta do informacji $x$, początkowo $0$
		\item[$A(x)$] Agent od którego została po raz ostatni otrzymana informacja równoznaczna z $x$
		
		\item[$U$] Parametr ufności
		
	\end{itemize}
	\subsection{Plotki}
	\begin{description}
		\item{Modyfikacja}
		\item{Przekazywanie}
	\end{description}
	\subsection{Logowanie}
	\section{Technologie}
	\section{Przykładowe eksperymenty}
	\section{Zadania/podział zadań}
\end{document}


