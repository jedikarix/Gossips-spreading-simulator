\documentclass{article}

\usepackage{fancyhdr}
\usepackage[english,polish]{babel}
\usepackage[utf8]{inputenc}
\usepackage{polski}
\usepackage{titling}
\usepackage{graphicx}

\title{Projekt WEDT + SAG \\ Symulator rozchodzenia się plotek \\ Dokumentacja wstępna}
\author{Karol Chęciński, Tomasz Gałecki, Jakub Łyskawa}
\setlength{\droptitle}{-10em}
\begin{document}
	\maketitle
	\section{Opis projektu}
		Projekt polega na utworzeniu symulatora rozchodzenia się plotek. Stosując uproszczony model kontaktów międzyludzkich oparty o sieć agentów porozumiewających się ze sobą, chcemy zbadać w jaki sposób rozprzestrzeniają się informacje w społecznościach. Agenty czerpią informacje z artykułów znalezionych w serwisach informacyjnych. Wybrane informacje przekazują dalej, do swoich sąsiadów w sieci połączeń. Zadaniem agenta po otrzymaniu plotki i przed rozpropagowaniem jej będzie ocena wiarygodności informacji. W czasie propagacji plotki istnieć będzie niewielka szansa, że informacja w niej zawarta zostanie zniekształcona. Model uwzględniać będzie także poziom zaufania między poszczególnymi agentami.
	\section{Opis modelu}
	\subsection{Graf połączeń}
	\begin{itemize}
		\item{Cechy grafu}
		\item{Elementy grafu}
		\item{Topologie}
	\end{itemize}
	\subsection{Charakterystyka agenta}
	Każdy węzeł w grafie połączeń reprezentuje agenta.
	\begin{description}
		\item[Interakcje między agentami]
		Agent może otrzymywać oraz przekazywać informacje innym agentom.
		Protokołem komunikacyjnym agentów jest język naturalny.
		\item[Stan agenta]
		Agent przechowuje otrzymane od innych informacje.
		Liczba tych informacji może być ograniczona,
		ponadto wykrywane są informacje które agent już posiada albo sprzeczne z posiadanymi.\\
		Ponadto dla każdej informacji oraz dla każdego połączonego agenta przechowywany jest poziom zaufania.
		\item[Model wiedzy]
		Otrzymane przez agenta informacje traktowane są jako jego dodatkowa wiedza,
		nakładająca się na pewną wiedzę początkową. 
		Od wiedzy zależy w jaki sposób zostaną przez agenta zmutowane informacje.\\
		Wiedza reprezentowana jest przy użyciu n-gramów,
		gdzie wiedza początkowa zostanie utworzona korzystając z pewnego korpusu,
		a n-gramy wynikające ze stanu agenta uwzględniane są z pewną wagą.
		\item[Parametry agenta] Każdy agent może być niezależnie parametryzowany następującymi wartościami:
		\begin{itemize}
			\item{Bazowe prawdopodobieństwo przekazania dalej otrzymanej informacji}
			\item{Maksymalna liczba przechowywanych informacji}
			\item{Bazowy wpływ informacji na wiedzę}
			\item{Parametry związane z modelem zaufania}
		\end{itemize}
	\end{description}
	\subsection{Model zaufania}
	\subsection{Plotki}
	\subsection{Logowanie}
	\section{Technologie}
	\section{Przykładowe eksperymenty}
	\section{Zadania/podział zadań}
\end{document}


